%% A simple template for a lab or course report using the Hagenberg setup
%% with the standard LaTeX 'report' class
%% äöüÄÖÜß  <-- keine deutschen Umlaute hier? UTF-faehigen Editor verwenden!

\documentclass[a4paper,english,11pt]{report}
%\documentclass[a4paper,ngerman,11pt]{report}

\usepackage{hgb}
\usepackage{hgbabbrev}
\usepackage{hgblistings}
\usepackage{hgbbib}
\usepackage{hgbheadings}

\RequirePackage[utf8]{inputenc}		% remove when using lualatex oder xelatex!

\graphicspath{{images/}}  % where are the images?
\bibliography{literatur}  % requires file 'literatur.bib'

\author{Sascha Zarhuber\\ S1510629021}
\title{IM790 Thesis Project 2\\ \emph{Event-based build pipeline for static content management}\\ Technical Documentation}
\date{\today}

%%%----------------------------------------------------------
\begin{document}
%%%----------------------------------------------------------
\maketitle
\tableofcontents
%%%----------------------------------------------------------

\chapter*{Abstract} % Vorwort

A modular, automated build pipeline API for static site generation.
Builds Metalsmith\footnote{\url{http://www.metalsmith.io/} -- Website of Metalsmith.} projects hosted on GitHub\footnote{\url{https://github.com} -- Website of GitHub.}, based on a well-formatted configuration file into downloadable tarballs.


%%%----------------------------------------------------------
\chapter{Installation}
%%%----------------------------------------------------------

\section{Prerequisites}
The following prerequisites are necessary for installing the REST API on a local machine:

\begin{itemize}
  \item Node.js\footnote{\url{https://nodejs.org/en/} -- Website of Node.js.} v6 and above
  \item MongoDB\footnote{\url{https://www.mongodb.com/} -- Website of MongoDB.} (local or remote)
  \item Mailgun\footnote{\url{https://www.mailgun.com/} -- Website of Mailgun.} API keys, and of course
  \item GitHub API keys.
  \item (optional) Yarn\footnote{\url{https://yarnpkg.com/en/} -- Website of Yarn.} (for faster installs)
\end{itemize}


\section{Setup}
To set up a working instance of this project, the following steps have to be done (in that order preferrably):

\begin{enumerate}
  \item \texttt{git clone https://github.com/saschazar21/contentcrane \&\& cd ./contentcrane}
  \item \texttt{yarn} or \texttt{npm install}
  \item \texttt{cp .env.sample .env}
  \item Add values to ENV keys in \texttt{.env} for providing system with necessary environment variables
\end{enumerate}
%

\section{Authentication}
This system uses OAuth 2.0\footnote{\url{https://oauth.net/2/} -- Website of OAuth 2.0} for user authentication. All API calls to the \texttt{/api/...} sub-route require an access token, issued by the built-in OAuth-service. Every token is valid for \emph{60 minutes} and cannot be renewed, instead, a new token must be obtained once the current one expires.


\section{User}
Since this service is tightly coupled with GitHub's services, every user has to sign up using the e-mail address he/she is also using for his/her GitHub account.\\
\textbf{CAUTION}: Since this is still a \emph{proof-of-concept}, a test user has to be set in \texttt{.env} for even being allowed to create a user.

\begin{center}
  \texttt{curl -X POST -L -u \{process.env.TEST\_USER\} -d "email=\{your github email\}\&password=\{your password\}" http://localhost:3000/api/user}
\end{center}


\section{Client}
For obtaining an access token, a user has to create a \emph{client}:

\begin{program}
  \caption{A standard response showing newly issued client data for use with the OAuth 2.0 authorization framework.}
  \label{prog:client_response}
  \begin{JsCode}
{
  "api_user":"{a hash string}",
  "api_secret":"{a longer hash string}"
}
  \end{JsCode}
\end{program}

\begin{center}
  \texttt{curl -X POST -L -u \{your user name\} http://localhost:3000/api/user/client}
\end{center}
After entering the user's correct password, the server should provide a response as shown in Program \ref{prog:client_response}.\\
\textbf{CAUTION}: Please be aware, that this information cannot be retrieved again -- once the client user/secret is lost, the user has to create a new client.

\section{Access Token}
After receiving client details, an \emph{access token} might be requested:

\begin{program}
  \caption{A standard response showing an issued access token as well as its expiry timestamp.}
  \label{prog:accesstoken_response}
  \begin{JsCode}
    {
      "access_token":"{access token hash}",
      "expires":"{expiry date, +60min in the future}",
      "token_type":"Bearer"
    }
  \end{JsCode}
\end{program}

\begin{center}
  \texttt{curl -X POST -L -u \{your client user\} -d "grant\_type=client\_credentials" http://localhost:3000/api/token}
\end{center}
After entering the client's correct password, the server should provide a response as shown in Program \ref{prog:accesstoken_response}.\\
\textbf{INFO}: Currently only the \emph{client\_credentials} grant type is implemented.


%%%----------------------------------------------------------
\chapter{API Endpoints}
%%%----------------------------------------------------------

Each API call must contain an \texttt{Authorization: Bearer \{access token hash\}} header, otherwise the response will be \texttt{403 Forbidden}.

\section{/api/project}
Creates a new entry for a project using a \emph{POST} request. Mandatory fields are:
\begin{description}
  \item[repo] -- The repository name on GitHub.
\end{description}

\section{/api/project/:owner/:repo/build}
Initiates a build job for the given repo using a \emph{POST} request. Optional fields are:
\begin{description}
  \item[force] -- If set to true, a full rebuild without the use of a cache is triggered.
  \item[base] -- A hash or date value to set as base; if date is used, always the first commit on that date is used as base.
  \item[head] -- A hash or date value to set as head; if date is used, always the first commit on that date is used as head.
\end{description}

\section{/api/project/:owner/:repo/status}
Gets the status of the latest build using a \emph{GET} request

\section{/api/project/:owner/:repo/download}
Downloads the latest successful build as \emph{.tar.gz} archive using a \emph{GET} request. Additional informations in response header are:
\begin{description}
  \item[X-Head] -- The commit hash value the archive was built from.
  \item[X-Timestamp] -- The timestamp as ISO string when the build succeeded.
\end{description}

\section{/api/project/:owner/:repo/delete}
Deletes the project from the database using a *POST* request. Optional fields are:
\begin{description}
  \item[recursive] -- When set to \emph{true}, also delete its builds from the database.
\end{description}

\chapter{Configuration}
Each project must contain either a \texttt{\_config.json} or \texttt{\_config.yml} file in its root directory. In that configuration file, the required settings for the Metalsmith API\footnote{\url{https://github.com/segmentio/metalsmith\#api} -- The Metalsmith API documentation.} should be placed after a \emph{global} property.
Below that, the configurations of needed Metalsmith plugins\footnote{\url{https://npms.io/search?q=metalsmith} -- A search of Metalsmith plugins on \emph{npms.io}.} may follow (see Prog. \ref{prog:configuration_sample}).\\
\textbf{CAUTION}: Please note, that every Metalsmith plugin name should explicitely be featured \emph{without} its \texttt{metalsmith-} prefix!

The same configuration as above also works as \emph{YAML}-formatted file saved as \texttt{\_config.yml}.

\begin{program}[t]
  \caption{A sample configuration structure for any project, which should be handled by the project's build pipeline.}
  \label{prog:configuration_sample}
  \begin{JsCode}
    {
      "global": {
        "source": "{source directory of your content}",
        "destination": "{output folder name}",
        "metadata": {
          "name": "Your global site name"
        }
      },
      "markdown": {
        "smartyPants": true
      }
    }
  \end{JsCode}
\end{program}

\chapter{Troubleshooting}
Please refer to the error message in the JSON object retrieved via\\
\texttt{/api/project/:owner/:repo/status}.
Generally it can be said, that as long as the build succeeds using the Metalsmith CLI\footnote{\url{https://github.com/segmentio/metalsmith\#cli} -- The Metalsmith CLI documentation.}, chances are high that it also succeeds using this service. Please note the similar structure of the configuration files.

%%%----------------------------------------------------------
%\MakeBibliography[nosplit]
%%%----------------------------------------------------------

\end{document}
