%% A simple template for a lab or course report using the Hagenberg setup
%% with the standard LaTeX 'report' class
%% äöüÄÖÜß  <-- keine deutschen Umlaute hier? UTF-faehigen Editor verwenden!

\documentclass[a4paper,english,11pt]{article}
%\documentclass[a4paper,twocolumn,english,11pt]{article}		
%\documentclass[a4paper,ngerman,11pt]{article}

\usepackage{hgb}
\usepackage{hgbabbrev}
\usepackage{hgblistings}
\usepackage{hgbbib}
\usepackage{hgbheadings}

\RequirePackage[utf8]{inputenc}		% remove when using lualatex oder xelatex!

\flushbottom

\graphicspath{{images/}}   % where are the images?
\bibliography{literatur}   % requires file 'literatur.bib'


\author{
Sascha Zarhuber\\ 
\texttt{sascha.zarhuber@students.fh-hagenberg.com}
}


\title{Selective rendering\\
in build pipelines}
\date{\today}

%%%----------------------------------------------------------
\begin{document}
%%%----------------------------------------------------------
\maketitle
\tableofcontents
%%%----------------------------------------------------------

\begin{abstract}
This document acts as the additional statement, enclosed to the Submission of the Agreement (\emph{Form B}). It should cover the current research progress and where I left off after the submission of the \texttt{Thesis Project 1}.\\
Furthermore, it contains already a preliminary table of contents, as well as an outlook of the literature that is going to be used during the ongoing research process.\\
The included Expose features an extension of the already submitted topic description.
\end{abstract}


\section{Working title}

The current working title for my Master's Thesis would be:

\begin{center}
\textbf{Selective rendering in build pipelines}
\end{center}

A former title, which appeared in earlier submissions, was \emph{Event-based build pipeline for static content management}. Due to the initial thoughts, and where the \emph{Thesis Project 1} has evolved to, a renaming of the description was necessary.\\
The new title should support me in improving the focus towards the important core parts of my research, which will be covered in the written Master's Thesis:
\begin{itemize}
	\item Find a selective approach, which distinguishes between \emph{modified} and \emph{unmodified} data parts.
	\item Separate both from each other.
	\item Apply rendering mechanism only on \emph{modified} data.
	\item Merge with previously rendered content to provide \texttt{iterative} versions of the respective website source.
\end{itemize}

\section{Expose}




%%%----------------------------------------------------------
  
\printbibliography  % alternatively: \MakeBibliography[nosplit]

%%%----------------------------------------------------------

\end{document}
