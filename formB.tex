%% A simple template for a lab or course report using the Hagenberg setup
%% with the standard LaTeX 'report' class
%% äöüÄÖÜß  <-- keine deutschen Umlaute hier? UTF-faehigen Editor verwenden!

\documentclass[a4paper,english,11pt]{article}
%\documentclass[a4paper,twocolumn,english,11pt]{article}		
%\documentclass[a4paper,ngerman,11pt]{article}

\usepackage{hgb}
\usepackage{hgbabbrev}
\usepackage{hgblistings}
\usepackage{hgbbib}
\usepackage{hgbheadings}

\RequirePackage[utf8]{inputenc}		% remove when using lualatex oder xelatex!

\flushbottom

\graphicspath{{images/}}   % where are the images?
\bibliography{literatur}   % requires file 'literatur.bib'


\author{
Sascha Zarhuber\\ 
\texttt{sascha.zarhuber@students.fh-hagenberg.com}
}


\title{Selective rendering\\
in build pipelines}
\date{\today}

%%%----------------------------------------------------------
\begin{document}
%%%----------------------------------------------------------
\maketitle
%\tableofcontents
%%%----------------------------------------------------------

\begin{abstract}
This document acts as the additional statement, enclosed to the Submission of the Agreement (\emph{Form B}). It should cover the current research progress and where I left off after the submission of the \texttt{Thesis Project 1}.\\
Furthermore, it contains already a preliminary table of contents, as well as an outlook of the literature that is going to be used during the ongoing research process.\\
The included Expose features an extension of the already submitted topic description.
\end{abstract}


\section{Working title}
The current working title for my Master's Thesis would be:

\begin{center}
\textbf{Selective rendering in build pipelines}
\end{center}

A former title, which appeared in earlier submissions, was \emph{Event-based build pipeline for static content management}. Due to the initial thoughts, and where the \emph{Thesis Project 1} has evolved to, a renaming of the description was necessary.\\
The new title should support me in improving the focus towards the important core parts of my research, which will be covered in the written Master's Thesis:
\begin{itemize}
	\item Find a selective approach, which distinguishes between \emph{modified} and \emph{unmodified} data parts.
	\item Separate both from each other.
	\item Apply rendering mechanism only on \emph{modified} data.
	\item Merge with previously rendered content to provide \texttt{incremental} versions of the respective website source.
\end{itemize}

\section{Expos\'{e}}
Due to my own experiences on the job, building and constantly rebuilding a static website easily slows down the whole workflow. In general, it can be said; \emph{the more files, the longer the build cycle}. Mostly, this is caused by the inclusion of files which do not require to be rebuilt individually.\\
Therefore, the main motivation behind my research culminates in ways to find a selective approach of only rendering files, which need to be rebuilt immediately for resulting in an up-to-date version of a given website.

\subsection{Possible research question}
The main research topic focuses on ways for a \emph{smart selection} of necessary files needed for an immediate content update on a static website, thus initiated by different source code modifications checked into a version control system (\texttt{VCS}), like \emph{Git}\footnote{Git: \url{https://git-scm.com}}, for example. Therefore, a possible research question would be the following:

\begin{center}
\textbf{How to speed up static site generation by a selective approach?}
\end{center}

The Thesis project mainly supports my research as a platform for testing different implementation approaches and as a testbed showing the speedup of the solution compared to a traditional system.

\subsection{Research goals}
\begin{itemize}
\item Find a selection method for altered files using git commits.
\item Separate \texttt{content} from \texttt{system build} files.
\item Filter modified sections in files, based on their diff (\emph{meta, body}).
\item Create \texttt{dependency-awareness} in the project tree (Enhance the rebuild focus on dependent files as well).
\end{itemize}

\section{Preliminary Table of Contents}
\begin{enumerate}
\item Introduction
\item State of the Art - Static Site Generators -- (\emph{3-5 pages})
\item Working with build pipelines -- (\emph{2-4 pages})
\item Version Control using \texttt{Git} -- (\emph{5-10 pages})

\begin{enumerate}
\item Collaboration on \emph{GitHub}\footnote{GitHub: \url{https://github.com}} -- (\emph{5-10 pages})
\end{enumerate}
\item File comparison

\begin{enumerate}
\item using \texttt{Diff} -- (\emph{5-10 pages})
\item other methods (metadata, hash values) -- (\emph{5-10 pages})
\end{enumerate}
\item The Selective Approach

\begin{enumerate}
\item Project structure  -- (\emph{4-6 pages})
\item Content/System file separation -- (\emph{3-6 pages})
\item Build pipeline with reduced load -- (\emph{4-6 pages})
\item Rendered result -- (\emph{3-5 pages})
\end{enumerate}
\item Usage in Thesis Project -- (\emph{3-5 pages})
\item Outlook: Continuous Integration -- (\emph{2-4 pages})
\end{enumerate}

\section{Milestones}
\begin{description}
\item [March 2017] -- Finishing \emph{Thesis Project}, have a working test setup ready for demonstrating the selective approach using \texttt{diff}.\\
Begin writing chapters \emph{2, 3}.
\item [April 2017] -- Chapters \emph{4, 5}. Refine file selection algorithm using research experiences. Perform unit tests, possibly under high load.
\item [May 2017] -- Chapter \emph{6}. Research/Tests on \texttt{problem-awareness} concerning project configuration.
\item [June 2017] -- Chapters \emph{7, 8} and submission of written Thesis.
\end{description}

\section{Relevant literature}
\begin{description}
\item [Creating Blogs with Jekyll] -- Vikram Dhillon\cite[]{dhillon2016}
\item [Version Control with Git] -- J Loeliger et al.\cite[]{loeliger2012version}
\item [Node.Js in Action] -- M. Cantelon et al.\cite[]{cantelon2017node}
\item [A Large Scale Study of Programming Languages and Code Quality in Github] -- Baishakhi Ray et al. \cite[]{ray2014github}
\item [An Algorithm for Differential File Comparison] -- J. W. Hunt et al.\cite[]{Hunt1975}
\item [Semantic Diff] -- D. Jackson et al.\cite[]{Jackson1994}
\end{description}

%%%----------------------------------------------------------
  
\MakeBibliography[nosplit]
%%%----------------------------------------------------------

\end{document}
