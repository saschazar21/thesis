\section{Solution strategies}
\label{sec:solution-strategies}

A primary task would be the focusing on critical issues to at least boosting the project's overall performance noticeably without losing too much track. This may be worked on in terms of collaboration, as well as in the project's setup, where on the one hand the team's performance and on the other hand the build engine's performance should be improved.

A lot of issues can be covered using GitHub's API, although some project specific adaptions are still necessary. Nevertheless, the API provides enough information for quickly perceiving a sufficient overview of the respective repository.


\subsection{Distributed development on GitHub}
\label{sec:solutions-distributeddevelopment}

Based on GitHub's API and the advantages of using Git as version control system, it is definitely a significant benefit to equip all project contributors with a GitHub account. While the public, open-source model is free of charge (see Sec. \ref{sec:github-history}), there are also different pricing models offered for privately held projects, which should be hidden to the public\footnote{\url{https://github.com/pricing} -- GitHub's pricing models.}.

Although dependent on financial expenditures, the additional value of working on a project with closed source (though it may be released as open source somewhere in the future) may be worth considering. Yet, full-featured access to the API is also included in the free tier though.

Not only GitHub offers a queryable API, also \emph{Bitbucket} provides an API with similar response data\footnote{\url{https://developer.atlassian.com/bitbucket/api/2/reference/} -- Bitbucket's API documentation.}, although certain features are missing, compared to GitHub. These missing features include for example certain project download functions, among others.

In addition to the API, GitHub's built-in In-Page Code Editor plays an important role for choosing it as core support tool for this project. Therefore, merging using pull requests and a continuous branching model for supporting staging versions qualify as proposed strategies to developers.


\subsection{Build cycles}
\label{sec:solutions-remotebuilding}

Supporting content authors in their workflow also means to not require them to install unnecessary build tools manually, unless critically needed. Due to the possibility of using GitHub's In-Page Editor, the whole Git checkout, commit and push process becomes in a way redundant too. Moreover, the online editor automatically creates pull requests on demand, so that the respective project owners should get notified automatically, if a merge is possible and therefore an update of the currently published project may be initiated.

Normally, a responsible user would pull the new state after a merge of the pull request, then execute the build pipeline. After the build process succeeded, he/she then has to take care for updating the webroot on the server, so that the newest version of the website gets delivered to the client upon request. However, this practice may easily get cumbersome, as the respective developer might get distracted by checking out new branches and possibly leaving behind his/her own work for the moment. Moreover, if the deployment has to be done manually, additional mistakes may happen during the whole action.

In this case, it makes sense to remotely outsource the build service and to possibly even automatically execute the render cycle. Used in conjunction with GitHub's \emph{Webhooks}, an external service would receive a build execution order via a HTTP POST request, based on certain predefined events in the GitHub repository \cite{GithubWebhooks}. Apart from the information the webhook provides, the service would even accept custom build options issued by responsible users, as the endpoint has to be publicly available anyhow. Once a build succeeds, the service should then notify a predefined list of users about the render cycle result and provide the outcome via download possibility.

\subsubsection{Existing remote services for static site generators}
\emph{CloudCannon}\footnote{\url{http://cloudcannon.com} -- CloudCannon, the Cloud CMS for Jekyll.} is probably the most popular online static site generator and offers a commercial external building service for Jekyll projects, together with source access using a GitHub or Bitbucket repository. According to its documentation, it currently supports Jekyll projects running v2.4.0 or newer\footnote{\url{https://docs.cloudcannon.com/building/versions/} -- Supported versions on CloudCannon documentation.}. It features a project file explorer and presents every new project as opinionated as Jekyll usually does (see Sec. \ref{sec:jekyll-technology}), as well as an automatic deployment service on their own subdomains. However, access to the rendered website files is not included, so every customer is dependent on using their hosting service.

\emph{BowTie}\footnote{\url{https://bowtie.io} -- Website of BowTie.} is similar to CloudCannon and is also offering a commercial online service. It seems to be a much more standalone service than CloudCannon, though it also offers GitHub integration, as well as custom Webhooks for event-based actions on external services.

%% Screenshot of Pancake push fail
\begin{figure} % h-ere, t-op, b-ottom, p-age
    \centering
    \includegraphics[width=0.75\textwidth]{remote-pancake.png}
    \caption{A screenshot of an approach to pushing a Jekyll project to \emph{Pancake}. As a result, the operation failed with an ``Unknown error''.}
    \label{fig:remote-pancake}
\end{figure}
%
\emph{Pancake}\footnote{\url{https://www.pancake.io} -- Website of Pancake.} is a free service for externally building static sites. It features an engine auto-detection and currently supports \emph{Jekyll, Wintersmith, Pelican, Sphinx, Hyde} and \emph{Middleman}\footnote{\url{https://github.com/pancakeio/detect/blob/master/heuristics.go} -- Currently supported static site generators by Pancake in raw source file on GitHub.}. Due to its non-commercial version, several restrictions are to be considered \cite{PancakeGitProjects}.

However, the service does currently not run stable, as an initial project setup failed (see Fig. \ref{fig:remote-pancake}). It seems that Pancake uses a post-receive hook for automatically trying to build a project, once it detected the underlying engine type. This causes waiting time for the developer on the one hand, but on the other hand informs whether a build was successful or failed. During another push attempt, it failed, as \emph{bundler}\footnote{\url{http://bundler.io} -- Bundler, a gem dependency manager.}, a gem dependency management tool required by Jekyll, was not mentioned in the ``Gemfile'' contained in the repository.

\subsection{Caching}
\label{sec:solutions-caching}

As stated before, most static site generators do not contain any form of caching mechanism by default -- if they do, caching is limited to the local machine a build is executed on. Since there is probably not an easy way of providing a form of remote caching, as this largely includes the necessity of external services to exchange a common status, as well as an index containing information about source and destination files for later rebuilds, it needs an equivalent strategy, which merely contains these information from a certain point in the past to the present, without relying on physical file structures to be exchanged.

Furthermore, such a caching strategy must be universally useable across all operating systems and ideally does not require any additional setup from the user. Moreover, it should also feature hassle-free integration into any project without depending on an external, yet unused service.

Keeping all of these issues under consideration, not every suggestion might get featured equally in the final solution -- the main reason is, that a kind of transformation like the one caused by a build pipeline always needs an existing status to build up from. So, tradeoffs are likely to accompany any form of decision to be made in this case.

\subsubsection{Caching based on diff}
As Git was chosen as version control system, diff is already part of the development suite. Therefore, a gapless detection of development progress between two arbitrary commits is possible. The diff format can be parsed to JSON and makes it easy for use in JavaScript. Thus, its usability for further processing on application level is assured\footnote{\url{https://runkit.com/saschazar21/diff-parsing-demo} -- An interactive example for fetching and parsing a diff-file.}.

The most important parts of a diff representation in this context are the file paths, as well as the type of modification on each file affected in the respective time span. Considering this kind of information, an existing repository might be quickly divided into unaffected and affected files -- where affected files, as well as their dependents possibly need to be selected for a rebuild. The final decision of the rebuild extent based on the diff, however, should be based on heuristics.

To conclude the consideration of using diff, the approach explained above is different from ``classical'' caching. Such a mechanism, founded on diff, is not dependent on support-files produced on its own (like a caching catalogue), but it requires a consistent and strict git workflow, otherwise it has no control over untracked files.
