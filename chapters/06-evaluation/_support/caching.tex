\section{Caching}
\label{sec:caching}

In terms of caching, the build pipeline is best evaluated when assuming the best possible, as well as the worst possible case. As already explained in ch. \ref{sec:challenges-cachedetermination} on p. \pageref{sec:challenges-cachedetermination}, such scenarios would be on the one hand a commit only containing content changes (e.g. new or modified blog posts) and on the other hand a commit containing a modification of the default template. Since the default template is very likely to act as a dependency of nearly all content files, a full rebuild is inevitable.

\subsection{Initial build}
An initial build is necessary every time a repository was registered using the REST API, or the repository's previous build attempts constantly failed and no successful outcome was produced yet. Not only caring for the required folder structure, a successful build cycle also provides information for a subsequent rendering process by storing its head commit hash value in the build log on the database. Any following build attempt is able to forge upon the last successful build files.

Therefore it is a good advice to have a successful initial build ready as soon as possible, as future build cycles profit from an early render history and a best possible caching structure. By omitting an early registration to the REST API, any initial build cycle in the future will last a significant amount of time longer, due to continuous progression which is not able to make use of any cached file structure.

%% compare initial registrations at different commit hashes


%% use for future subsection
Thereby it is not important to check for any existing file structure, if any previously failed build attempt forced a rebuild due to its commit history, as the current build also has to use the same commit history, thus also forcing a rebuild.
