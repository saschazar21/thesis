\subsection{Express.js for REST}
\label{sec:foundation-express}
Starting with Express.js, the sample code in chapter \ref{sec:primarythoughts-restapi} on p. \pageref{sec:primarythoughts-restapi} shows a reasonable example on how to easily provide an API endpoint. While the example only returns a string containing ``Hello World!'', a JSON structure may also be used and is probably a better choice for working programmatically on the response data later on.

Furthermore, a good advice would be to use a modular form of route definitions, since the main source file will soon get too bloated and may grow a lot of spaghetti-code in it. This may be achieved in outsourcing the routes in specific files and/or folders and importing them via a \emph{require}-statement. As a bonus, an external source file containing route definitions also allows for custom logic and middlewares, which may be hidden to the rest of the application by default \cite[p. 220f]{cantelon2017node}.

\subsubsection{Middleware}
Especially when depending on advanced application logic (e.g. user authentication, database management, etc\ldots), further tasks containing validation checks or user definitions may get necessary. If these tasks are required by more than one route, it makes sense to abstract their logic into reusable components for use as middleware in these specific routes \cite[223]{cantelon2017node}. Optionally, more than one middleware may be used on a single route, where their placement stands for their execution order -- from first to last.

\lstinputlisting[label={list:express-middleware}, language=JavaScript, caption={An example for middleware ordering, where \emph{firstMiddleware} gets called right before \emph{secondMiddleware}. Both middlewares have to succeed (e.g. return \emph{done}-callback function) in order to grant access to the ``/secret'' route.}]{chapters/05-implementation/_support/middleware.js}

\subsubsection{OAuth 2.0}
\label{sec:foundation-express-oauth}
\emph{OAuth 2.0} stands for an open authorization framework, which grants limited access to a certain HTTP service, either on behalf of a resource owner (e.g. allow access to user data of a social network account), or by allowing a third-party application to obtain access on its own behalf \cite[1]{hardt2012oauth}. In this case, the latter is more interesting, as a programmatical access may be achieved by issuing an access token via a ``client credentials'' grant type. Therefore, an application-only access is possible without depending on any user interaction.

The whole authentication process is necessary, as the final web application will hold different user accounts, as well as their registered projects. Thus, every client (human or non-human) may interact with the application's API only via certain issued tokens, which ideally are only valid for a specific amount of time before they expire \cite[43]{hardt2012oauth}.
