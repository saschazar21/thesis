\section{Hexo}
\label{sec:hexo}

\texttt{Hexo} understands itself as counterpart to \texttt{Jekyll}, mostly by covering the same ideas of static site generation, but building up completely on \texttt{Node.js}. It even offers a migration service for \texttt{Octopress}- and \texttt{Jekyll}-users who are willing to switch.

\subsection{History}
\label{sec:hexo-history}
\texttt{v1.0.0} was originally released in March 2013\footnote{\url{https://github.com/hexojs/hexo/releases/tag/1.0.0} -- Hexo v1.0.0 release page on \texttt{GitHub}.}, although development on \texttt{GitHub} dates back to September 2012 as the first commit was published using the message \emph{``init''}.

\emph{Tommy Chen}, its creator, first used \texttt{Octopress}\footnote{\url{http://octopress.org} -- Octopress website.} but quickly became dissatisfied with its performance, as the rendering of 54 blog posts already took more than a minute of compile time \cite{Chen2012hexodebut}. Since he assumed \texttt{Ruby} might be the cause for the lack of performance of his primarily used blogging framework, and further development on this case was not likely to happen any time soon, he decided to look for something which got his attention shortly before: \texttt{Node.js}.

However, \texttt{Node.js} was not really a big player back at that time, so the offer of blogging frameworks written in \texttt{JavaScript} was very dense and not really fitting the needs of \emph{Tommy Chen}. In his announcement article for \texttt{Hexo} \cite{Chen2012hexodebut}, he references a blog post of \emph{Boris Mann}, also an \texttt{Octopress} user at that time, listing a few \texttt{Node.js}-based blogging frameworks, which were already around in June 2012\footnote{\url{http://blog.bmannconsulting.com/node-static-site-generators} -- Blog article of \emph{Boris Mann} about \texttt{Node.js}-based blogging frameworks.}. Interestingly, only two of all the mentioned ones, \texttt{Wintersmith} and \texttt{DocPad} are still actively maintained today.

\subsection{Technology}
\label{sec:hexo-technology}
As already stated above, \texttt{Hexo} primarily consists of \texttt{JavaScript}, thus making it easier to start for developers with a front-end web development background. In fact, its \texttt{GitHub} repository shows \texttt{JavaScript} holding a share of 100\% on the source code\footnote{\url{https://github.com/hexojs/hexo} -- Hexo repository on \texttt{GitHub}.}.

\subsubsection{Advantages}
Right from the start, \texttt{Hexo} presents itself using its feature-rich command-line interface (\emph{CLI}), similar to \texttt{Jekyll}. Once it is installed, \texttt{hexo init my\_project} scaffolds a new starter template into the \texttt{./my\_project} folder.\\
As \emph{Tommy Chen} himself wanted an easy-to-use replacement for \texttt{Octopress}, \texttt{Jekyll} and \texttt{Hexo} share a lot of common features; their content files use both \texttt{YAML} frontmatter and \texttt{Markdown} by default, even the main configuration file uses a very similar structure in both frameworks. This should make it extremely easy switching from \texttt{Jekyll} to \texttt{Hexo}.

First and foremost, programming-unaware content authors might especially like its CLI, as it also offers to create files based on the \texttt{hexo new} command. Depending on other submitted command-line arguments, \texttt{Hexo} may automatically put the new file in the according sub-folder, whether it is a \emph{draft, page} or \emph{post}. Publishing a draft is as easy as \texttt{hexo publish}.\\
When creating content, \texttt{Hexo} also contains a feature-rich, \texttt{Octopress}-inspired custom tag selection for including content from \emph{YouTube, Vimeo} or \emph{GitHub Gists}.

Additionally, its plugin collection is also constantly growing and mostly community supported. A special naming convention using \texttt{hexo-} as prefix helps by determining which plugins to auto-load out of the \texttt{node\_modules} folder. Using this way, \texttt{Ruby's} \emph{convention over configuration} mantra is ported to \texttt{JavaScript} as well and especially supports beginners by not having to define the usage of a certain plugin.

\subsubsection{Disadvantages}
\texttt{Hexo} might look like as an ideal replacement for \texttt{Jekyll}, but since both share so much similarities, they also share some disadvantages. Whereas \texttt{Jekyll} ships with \emph{Liquid} and \emph{Sass} as standard, \texttt{Hexo} does with \emph{EJS} and \emph{Stylus}.\\
Although clearly stated, that both of these plugins might be easily uninstalled later on\footnote{\url{https://hexo.io/docs/setup.html\#package-json} -- Hexo's setup documentation.}, the whole setup pre-installation seems as opinionated as \texttt{Jekyll's}.

In addition to the already mentioned plugin system, a missing configuration option might as well turn out to be misleading in terms of customization options, especially when being dependent on the CLI. If customization is necessary, the developer often is forced to switch to the \texttt{JavaScript} API\footnote{\url{https://hexo.io/api/} -- Hexo's JavaScript API documentation.} or to add a plugin to the project to make the build pipeline fit the customization's needs.

When it comes to caching, \texttt{Hexo} uses a homebrew version of \emph{JSON memory caching} called \texttt{Warehouse}, also created by \emph{Tommy Chen}\footnote{\url{https://github.com/tommy351/warehouse} -- Warehouse repository on \texttt{GitHub}.}, initially mentioned in the release notes of \texttt{3.2.0-beta.2} \cite{Chen2015hexorelease}. Using this plugin, a mode called ``Hot processing'' should enable faster re-builds. The main drawback here might be the caching speed, which is on the one hand filling up the memory when working on bigger projects, whereas the persisting of the database is fully dependent on file input/output write speeds of the underlying hard disk.\\
Furthermore, a constantly growing database file is hardly transferrable when trying to implement a decentralized building system out of \texttt{Hexo}.
