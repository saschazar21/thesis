\section{Metalsmith}
\label{sec:metalsmith}

Compared to the already described static site generators, \texttt{Metalsmith} is to be considered the youngest project.
It might also be the most radical project, as it was designed to consist of \textbf{nothing but plugins}. Therefore, in terms of still being a static site generator, it tries hard to push the limits much further than previously mentioned \texttt{Jekyll} and \texttt{Hexo}.

\subsection{History}
\label{sec:metalsmith-history}
Initially developed by \emph{Segment}\footnote{\url{https://segment.com} -- Segment's website.} for their internal needs, such as \emph{documentation, help} and \emph{blog pages} \cite{Metalsmith2015buildingblocks}, \texttt{Metalsmith} was finally open-sourced and made publicly available around February 2015 -- its commit history on \texttt{GitHub} dates back to February 4\textsuperscript{th}, 2014.\\
Most of the commits at that time were published by \emph{Ian Storm Taylor}, co-founder of \emph{Segment}, although his contribution to the project ends after releasing \texttt{v2.1.0} on September 24\textsuperscript{th}, 2015\footnote{\url{https://github.com/segmentio/metalsmith/commits/master?author=ianstormtaylor} -- Contributions of \emph{Ian Storm Taylor} to the Metalsmith repository on \texttt{GitHub}.} at the moment.

Like \texttt{Hexo}, \texttt{Metalsmith's} repository completely consists of \texttt{JavaScript} code, as its developers also were unsatisfied with the then existing static site generators. According to \emph{Chris Sperandio}, the \texttt{Metalsmith} developers desired pure flexibility for their ``wide array of use cases'', while other frameworks all asked for a certain structure on the content \cite{Metalsmith2015buildingblocks}.

\subsection{Technology}
\label{sec:metalsmith-technology}
Since \texttt{Metalsmith} consists of only plugins, specifically written for this very framework, there is no real standard setup provided. Although there are a few tutorials and best practices listed in its \texttt{GitHub} repository\footnote{\url{https://github.com/segmentio/metalsmith\#the-secret} -- Possibilities for using Metalsmith.}, as well as in a repository called ``\emph{awesome-metalsmith}''\footnote{\url{https://github.com/metalsmith/awesome-metalsmith} -- ``Awesome'' Metalsmith resources list.}, the initial dive-in might scare a few people away, since \texttt{Metalsmith} might not be as well documented as the previously mentioned frameworks. Moreover, most developers seem to experience a very steep learning curve at first.

\subsubsection{Advantages}
Every developer is able to shape \texttt{Metalsmith} exactly to his/her needs, once he knows about the basic usage. It ships with a CLI, as well as a \texttt{JavaScript} API, where the ``real hacking'' is possible.\\
The CLI gets easily configured via a \texttt{metalsmith.json} file, stored in the project directory. It consists mainly of general project configurations, placed in the object's root, as well as an array of used plugins, respectively combined with their configuration.

It neither contains a pre-installed template engine, nor any other pre-processing tools, like \texttt{Sass}, or \texttt{Less}. However, the available plugins support most of them to a satisfying extent, so that the \texttt{metalsmith-layouts} plugin is a wrapper for \texttt{consolidate.js}, which per se acts as a wrapper for the most common template engines\footnote{\url{https://github.com/tj/consolidate.js\#supported-template-engines} -- Consolidate.js-supported template engines on \texttt{GitHub}.}. Therefore, the developer is able to select the tools based on his/her preferences and may initialize a project from scratch, without needing to clean up any pre-installed demonstration files first.

Using the built-in \texttt{JavaScript} API, it is also possible to invoke the needed modules programmatically, which is one of the core topics of this Thesis.

\subsubsection{Disadvantages}
So much freedom in designing a project may also cause some dangers. In this case, one of the most crucial things is the arrangement of plugins in the configuration. Since \texttt{Metalsmith} acts as a streaming build system, every transformation of the content must happen at its time. This is especially important when a plugin might alter the underlying code in a way, that a following plugin becomes useless, as it might not be able to succeed in its predefined task.\\
As an example, the following code snippet shows one bold example of misconfiguration (See line no. 16):

%% Update line no. above, if code changes
\lstinputlisting[language=JavaScript]{chapters/02-state-of-the-art/_support/metalsmith.js}
