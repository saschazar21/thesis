\section{Jekyll}
\label{sec:jekyll}

As already described in section <<\emph{\nameref{par:creatingcontent}}>> on p. \pageref{par:creatingcontent}, \texttt{Jekyll} was created out of the need for avoiding to service the blogging engine before writing and publishing content. Since it is deeply integrated into \texttt{GitHub}, it is considered as the probably most-popular static site generator.

\subsection{History}
\label{sec:jekyll-history}
\emph{Tom Preston-Werner}, co-founder of \texttt{GitHub}\footnote{\url{https://github.com} -- GitHub Inc.}, announced it in October 2008 in one of his blog posts \cite{PrestonWerner2008jekyll}. Already in December 2008, it was introduced as build engine for the then newly featured \textbf{GitHub Pages} service, allowing owners of repositories to publish a static website by just pushing to a certain \texttt{master} or \texttt{gh-pages} branch \cite{PrestonWerner2008githubpages}, which is still available for free to this day.\\
All of this happened just 6 (respectively 8) months after \texttt{GitHub} was launched \cite{PrestonWerner2008githublaunch} and is now even being used by technology-leading companies to showcase their Open-Source efforts\footnote{\url{https://github.com/showcases/github-pages-examples} -- GitHub Pages examples.}.

\subsection{Technology}
\label{sec:jekyll-technology}
\texttt{Jekyll} was entirely written in \textbf{Ruby}, as \emph{Tom Preston-Werner} rather saw himself as a software developer in the first place, than as a content author \cite{PrestonWerner2008jekyll}. Until now, the repository for \texttt{Jekyll} still consists mainly of \texttt{Ruby} code at a share of roughly 77.5\% (See Fig. \ref{fig:jekyll-languages}).

% Jekyll programming languages graphic
\begin{figure}
    \centering
    \includegraphics[width=0.9\textwidth]{jekyll-repo-languages.png}
    \caption{A screenshot taken from the \texttt{GitHub} repository page, showing the spread of \textbf{programming languages} used for developing \texttt{Jekyll}. \textbf{Ruby} acts as the most prominent language, holding a share of roughly \textbf{77.5\%}.}
    \label{fig:jekyll-languages}
\end{figure}
%

\subsubsection{Advantages}
One of the main advantages is the modular structure of its code base. By inheriting different \texttt{Ruby} classes, it is quite easy to extend and add features to fit the developer's needs. Due to its wide-spread usage initiated through the \texttt{GitHub} universe, \texttt{Jekyll} also has an accordingly huge user base and is therefore well documented \cite[26]{dhillon2016}.\\ Furthermore, its website\footnote{\url{http://jekyllrb.com} -- Jekyll website.}, which mainly acts as starting basis for documentation, is not only available as open-sourced git repository, it is also built using \texttt{Jekyll} to prove its universality.

Starting from scratch, the command \texttt{jekyll new my\_project} installs a blog environment for starters in the \texttt{./my\_project} folder. The basic install consists of an elementar blog post structure, \texttt{sass} source files, and a few template files written for Shopify's \texttt{Liquid}\footnote{\url{https://help.shopify.com/themes/liquid} -- Shopify's Liquid template engine.} engine.\\
Using this starting environment, the unexperienced developer quickly gets a sufficient overview of what is generally possible using \texttt{Jekyll}, whereas the content author is able to fully concentrate himself on writing content, as the used \texttt{Markdown} markup language requires little to no prior syntax knowledge.

Furthermore, \texttt{Jekyll} already ships with a built-in webserver for quickly reviewing the rendered static output.

\subsubsection{Disadvantages}
As powerful as \texttt{Ruby} might be designed, many unskilled developers are facing difficulties right from the beginning, as most of them experience a steep learning curve. Nearly every single bit of customizing \texttt{Jekyll} requires \texttt{Ruby} knowledge, especially if it is desired to move along the ``predefined'' way and not including third-party extensions like \texttt{Node.js} tools or else.

Additionally, its template language, \texttt{Liquid}, offers customization on a very high level, so it might happen to confuse core application logic\footnote{How \texttt{Jekyll} processes data into programmatically readable structures.} with template logic\footnote{How \texttt{Liquid} transforms these structures into browser-readable HTML.}. To make things worse, different template constructions might also evolve over time and therefore causing a parallel coding universe when trying to surpass difficulties in the core application logic.

%% bring some difficulties concerning ruby versions
%% ruby gems
