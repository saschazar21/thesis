\subsection{Frontmatter}
\label{sec:buildpipelines-frontmatter}

\begin{program}
  \caption{frontmatter.md}
  \label{list:frontmatter}
\lstinputlisting{chapters/03-technical-foundations/_support/frontmatter.md}
\end{program}

Program \ref{list:frontmatter} shows a sample usage of frontmatter inside a Markdown file. Bounded by three dashes above the main content source, it allows certain per-file metadata definitions, which will be parsed at build time and provided for the template rendering engine.
% Mention something about omitting databases using frontmatter meta

As an example, the selected template for this sample file may also hold a list for the mentioned \emph{tags}, as well as a placeholder for the \emph{author}'s name and/or \emph{title}. The main content gets then rendered into the respective placeholding tag, already self-containing a basic structure.

Using a \emph{template: false} declaration, some plugins may prevent rendering the content into a template. This might be interesting in cases, where different partials should be included in the DOM by some sort of asynchronous JavaScript later on.
