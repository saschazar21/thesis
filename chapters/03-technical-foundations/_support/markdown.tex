\subsection{Markdown}
\label{sec:buildpipelines-markdown}

\lstinputlisting[caption={markdown.md}, label={list:markdown-demo}]{chapters/03-technical-foundations/_support/markdown.md}

\texttt{Markdown} consists of shorthand conventions, which should be easier to type for content creators  \cite[38]{dhillon2016}.
Therefore, it makes understanding \texttt{HTML} not a necessary precondition anymore, as a basic content structure may be easily achieved when prepending/surrounding text with certain special characters like \texttt{\#, *, \_, \ldots} (See Listing \ref{list:markdown-demo}).

Originally created by \emph{John Gruber}\footnote{\url{http://daringfireball.net/projects/markdown/} -- Markdown website.} as a plugin for \emph{Movable Type} and \emph{Blosxom} blogging engines \cite{Markdown2004introduction} in March 2004, it should be a supportive tool for users against the complexity of formal markup languages (e.g. \texttt{HTML5}) \cite[4]{RFC7764}. According to \emph{Gruber's} intention, there is no ``invalid'' \texttt{Markdown}, as he suggests the author should either ``keep on experimenting'' or ``change the processor'', if the output happens to fail his/her expectations \cite[5]{RFC7764}.

\texttt{GitHub} finally adapted \texttt{Markdown} to its own version, called ``GitHub Flavored Markdown'' (\emph{GFM}). The people behind it enhanced its original functions to also support \emph{code blocks, tables, strike-through text}, as well as \emph{auto-linking} URL structures in the content \cite[18]{RFC7764}. Additionally, also \texttt{GitHub}-specific functions, such as \emph{user mentions, commit references} or \emph{emojis}, may be used\footnote{\url{https://guides.github.com/features/mastering-markdown/\#GitHub-flavored-markdown} -- GitHub Flavored Markdown cheat sheet on \texttt{GitHub}.}.

Since then, \texttt{GitHub} renders browser-friendly versions of general descriptions written in \texttt{Markdown}, for providing fast and easy overviews of the respective repository. As an example, an existing \texttt{README.md} always appears below the root file tree section on a repository front page \cite[5]{gandrud2013github}.
