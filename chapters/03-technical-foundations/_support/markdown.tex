\subsection{Markdown}
\label{sec:buildpipelines-markdown}

\lstinputlisting[caption={markdown.md}, label={list:markdown-demo}]{chapters/03-technical-foundations/_support/markdown.md}

\texttt{Markdown} consists of shorthand conventions, which should be easier to type for content creators  \cite[38]{dhillon2016}.
Therefore, it makes understanding \texttt{HTML} not a necessary precondition anymore, as a basic content structure may be easily achieved when prepending/surrounding text with certain special characters like \texttt{\#, *, \_, \ldots} (See Listing \ref{list:markdown-demo}).

Originally created by \emph{John Gruber}\footnote{\url{http://daringfireball.net/projects/markdown/} -- Markdown website.} as a plugin for \emph{Movable Type} and \emph{Blosxom} blogging engines \cite{Markdown2004introduction} in March 2004, \texttt{GitHub} adapted it to its own version, called ``GitHub Flavoured Markdown'' (\emph{GFM}). The people behind it enhanced its original functions to also support % Yeah, right... what does it also support??

Since then, \texttt{GitHub} renders browser-friendly versions of general descriptions written in \texttt{Markdown}, for providing fast and easy overviews of the respective repository. As an example, an existing \texttt{README.md} always appears below the root file tree section on a repository front page.
