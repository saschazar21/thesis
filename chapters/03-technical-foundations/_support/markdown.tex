\subsection{Markdown}
\label{sec:buildpipelines-markdown}

\lstinputlisting[caption={markdown.md}, label={list:markdown-demo}]{chapters/03-technical-foundations/_support/markdown.md}

Markdown consists of shorthand conventions, which should be easier to type for content creators  \cite[38]{dhillon2016}.
Therefore, it makes understanding HTML not a necessary precondition anymore, as a basic content structure may be easily achieved when prepending/surrounding text with certain special characters like \texttt{\#, *, \_, \ldots} (See Listing \ref{list:markdown-demo}).

Originally created by \emph{John Gruber} as a plugin for \emph{Movable Type} and \emph{Blosxom} blogging engines in March 2004 \cite{Markdown2004introduction}\cite{Markdown2004main}, it should be a supportive tool for users against the complexity of formal markup languages (e.g. HTML5) \cite[4]{RFC7764}. According to Gruber's intention, there is no ``invalid'' \texttt{Markdown}, as he suggests the author should either ``keep on experimenting'' or ``change the processor'', if the output happens to fail his/her expectations \cite[5]{RFC7764}.

GitHub finally adapted Markdown to its own version, called ``GitHub Flavored Markdown'' (\emph{GFM}), somewhere around April 2009\footnote{\url{https://github.com/mojombo/github-flavored-markdown/issues/1} -- GitHub Flavored Markdown examples by Tom Preston-Werner.}. The people behind it enhanced its original functions to also support \emph{code blocks, tables, strike-through text}, as well as \emph{auto-linking} URL structures within the content \cite[18]{RFC7764}. Additionally, also GitHub-specific functions, such as \emph{user mentions, commit references} or \emph{emojis}, may be used \cite{GithubFlavoredMarkdown}.

Since then, GitHub renders browser-friendly versions of general descriptions written in Markdown, for providing fast and easy overviews of the respective repository. As an example, an existing \texttt{README.md} always appears below the root file tree section on a repository front page \cite[5]{gandrud2013github}.
