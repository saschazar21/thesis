\section{Git}
\label{sec:git}

Today, hardly any software project is started without any form of version control system (\texttt{VCS}). It supports developers as a back-up system and living archive of their work, as data generally is ephemeral and can be lost easily \cite[1]{loeliger2012version}. Although there are many different variations offered, some of the most popular today are \emph{Git, Apache Subversion, Mercurial and Bazaar}.

\subsection{History}
\label{sec:git-history}
\texttt{Git} was initially published by \emph{Linus Torvalds} on April 7\textsuperscript{th}, 2005 \cite[6]{loeliger2012version}. This was necessary, as the ``free'' version of the then used VCS for the \emph{Linux} kernel development, \texttt{BitKeeper}, was restricted in a way it was not suitable any longer for the community behind it. Furthermore, the search of an already available alternative to the \texttt{BitKeeper} system failed due to an unsatisfying combination of needed features, so \emph{Torvalds} came up with his own VCS flavor, containing all desired functionalities for further \emph{Linux} development (among others) \cite[4]{loeliger2012version}:

\begin{itemize}
  \item Distributed development
  \item Handle thousands of developers
  \item Efficient performance
  \item Support branched development
  \item Free, as in freedom
\end{itemize}

%% Write something, why it is so famous. distributed development - other use cases .. blargh.
