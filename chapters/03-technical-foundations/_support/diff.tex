\section{Diff}
\label{sec:diff}

``\emph{diff} reports file differences between two files, expressed as a minimal list of line changes (\ldots)'' \cite[1]{Hunt1976}. Existing more than 40 years now, it has been an essential tool for file comparison throughout the history of computing -- furthermore, it is also a core component of Git, which contains its own version called \emph{git diff} \cite[108]{loeliger2012version}.

\subsection{History}
\label{sec:diff-history}
Initially published by \emph{James W. Hunt} and \emph{Malcolm D. McIlroy} in July 1976 when working at \emph{Bell Labs}, the algorithm was later used in \emph{UNIX} as application called diff. \emph{Paul Eggert} and \emph{Richard Stallman} (among others) also wrote the diff application as part of their \emph{GNU diffutils}\footnote{\url{http://manpages.ubuntu.com/manpages/zesty/en/man1/diff.1.html} -- Manpage for GNU diff.}, which is nowadays mainly distributed in Linux derivatives, MacOS, as well as part of Git. They used an improved algorithm published by \emph{Webb Miller} and \emph{Eugene W. Myers} in 1985 \cite[3]{mackenzie2003comparing}, who proved that the original ``Hunt-McIlroy algorithm'' is inefficient on certain special cases. As a test case, they used a file containing 1000 blank lines, and a second file, consisting of the initial file, but containing a single non-blank file on both ends. As a fact, using other experiments performed on typical files, Miller's and Myers' algorithm ran roughly four times faster \cite[p. 1034f]{miller1985file}.

\subsection{Technology}
diff's core task is finding the ``shortest sequence of insertions and deletions that will convert the first string to the second'' \cite[1025]{miller1985file} together with finding the longest common subsequence occurring in both files \cite[2]{Hunt1976}. Combined with the mathematical algorithm, it should provide an easily understandable format for humans, consisting of line numbers joined with \emph{a, c} or \emph{d} (append, change, delete), as well as \emph{<} and \emph{>} line prefixes, showing the affiliation either to the initial or compared file. This is called the ``Normal Format'' \cite[12]{mackenzie2003comparing}. Program \ref{list:diff-normalformat} shows a sample output, comparing the strings \texttt{a b c d e f g} and \texttt{w a b x y z e} (one line per letter) \cite[p. 1f]{Hunt1976}.

\label{sec:diff-technology}
\begin{program}
  \caption{sample.diff}
  \label{list:diff-normalformat}
  \begin{GenericCode}
  0a1
  > w
  3,4c4,6
  < c
  < d
  ---
  > x
  > y
  > z
  6,7d7
  < f
  < g
  \end{GenericCode}
\end{program}

% TODO: Ugly, remove if not necessary anymore, cares for vertical spaces above subsections
% \vspace{20pt}

\subsubsection{The Unified Format}
To provide a more readable user experience, GNU diff contains an improved format, called Unified Format, removing redundancy by using a more compact syntax. It can be selected as output format by executing diff together with a \texttt{-u} flag \cite[16]{mackenzie2003comparing}, whereas git diff uses this as standard format to show changes within the current working tree \cite{GitDiff}.

\begin{program}
  \caption{unified\_format.diff}
  \label{list:diff-unifiedformat}
  \begin{GenericCode}
  --- oldfile	2017-04-13 09:42:47.474769553 +0200
  +++ newfile	2017-04-13 09:43:13.898566935 +0200
  @@ -1,7 +1,7 @@
  +w
   a
   b
  -c
  -d
  +x
  +y
  +z
   e
  -f
  -g
  \end{GenericCode}
\end{program}

As an example, Program \ref{list:diff-unifiedformat} shows the same diff output as Program \ref{list:diff-normalformat}, only as Unified Format using the following components: \texttt{------ \{filename\} \{timestamp\}} indicates the initial file together with the timestamp it was created, whereas \texttt{+++ \{filename\} \{timestamp\}} is the same as above, but for the compared file.
\texttt{@@ -\{intial line range\} +\{compared line range\} @@} shows the affected line range, where \texttt{-1,7} indicates the following 7 lines, starting from the first line of the initial file. Lastly \texttt{+} marks a line as added in compared file and \texttt{-} marks a line as deleted in the compared file. To make the above explanation a little bit more clear, an additional example with a text speaking for itself is shown in Program \ref{list:diff-explanation}.

\begin{program}
  \caption{file.diff}
  \label{list:diff-explanation}
\lstinputlisting{chapters/03-technical-foundations/_support/file.txt}
\end{program}

It can be clearly seen, that line 3 shows a growth of \emph{newfile} by two lines: \texttt{-1,8} vs. \texttt{+1,10}. Having also a color-coded representation, it would boost the readability of such diff outputs once again.


% TODO: Ugly, remove if not necessary anymore, cares for vertical spaces above subsections
%\vspace{20pt}
\subsubsection{Usage with Git}
As already stated, diff is one of the core components of Git. Not only does it support determining changes in the source code between a snapshot and another, it may also reveal merge conflicts, if segments are mutually exclusive and therefore preventing a flawless propagation of development. Thus, a varying development history of different origins (e.g. branches) not compatible to each other might be indicated. Furthermore, a conflict may also happen, if a developer forgot to \emph{pull} the latest changes before committing his/her current development progress. These conflicts may only be handled through human guidance \cite[124]{loeliger2012version}.

\begin{program}
  \caption{A snippet of a file called ``manual.txt'', which is affected by a conflict. Content between \texttt{HEAD} and \texttt{=======} contains the local version, content below contains the foreign conflicting version.}
  \label{list:diff-conflict}
\begin{GenericCode}
<<<<<<< HEAD:manual.txt
I (the developer) am right!
=======
The branch_name is right!
>>>>>>> branch_name:manual.txt
\end{GenericCode}
\end{program}

A conflict presents itself primarily through a message similar to:\\
\texttt{CONFLICT (content): Merge conflict in file\\
Automatic merge failed; fix conflicts and then commit the result.}\\
If anything like the above happens, the affected files by the conflict also contain a structure like shown in Program \ref{list:diff-conflict}. A conflict can then be resolved by removing its markers and picking the appropriate resolution of either side of the \texttt{=======} delimiter \cite{GitConflicts}, as well as mixing them to the developers needs \cite[126]{loeliger2012version}. A single file may also contain multiple conflicts.

\subsubsection{Usage with GitHub}
Especially when interacting with GitHub's REST API, it is very easy to generate and import file diffs of a given repository. Whether two branches or two commits using their \emph{SHA} values are compared, a single HTTP request suffices for programmatically retrieving data, which is normally only accessible using a terminal emulator.

Depending on the requested \emph{media type} in the appropriate HTTP header field, either a full-featured diff, patch or JSON containing per-file patches is emitted by the API. If the latter was used, the underlying diff is translated into a JSON object, containing information like the number of additions, deletions and changes, as well as the mentioned \emph{patch} for each file.

As a consequence, a repository does not necessarily have to be \emph{cloned}, as it may be patched constantly using the API to keep it up to date -- using this method, patch is even able to create and delete files, if necessary \cite[57]{mackenzie2003comparing}. The only prerequisite is to keep track of the single commit hashes the patches are applied from.
