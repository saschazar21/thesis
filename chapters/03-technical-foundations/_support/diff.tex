\section{Diff}
\label{sec:diff}

``\texttt{diff} reports file differences between two files, expressed as a minimal list of line changes (\ldots)'' \cite[1]{Hunt1976}. Existing more than 40 years now, it has been an essential tool for file comparison throughout the history of computing -- furthermore, it is also a core component of \texttt{Git}, which contains its own version called \texttt{git diff} \cite[108]{loeliger2012version}.

\subsection{History}
\label{sec:diff-history}
Initially published by \emph{James W. Hunt} and \emph{Malcolm D. McIlroy} in July 1976 when working at \emph{Bell Labs}, the algorithm was later used in \emph{UNIX} as application called \texttt{diff}. \emph{Paul Eggert} and \emph{Richard Stallman} (among others) also wrote the \texttt{diff} application as part of their \emph{GNU diffutils}\footnote{\url{http://manpages.ubuntu.com/manpages/zesty/en/man1/diff.1.html} -- Manpage for \emph{GNU diff}.}, which is nowadays mainly distributed in \emph{Linux} derivatives, \emph{MacOS}, as well as part of \texttt{Git}. They used an improved algorithm published by \emph{Webb Miller} and \emph{Eugene W. Myers} in 1985 \cite[3]{mackenzie2003comparing}, who proved, that the original ``Hunt-McIlroy algorithm'' is inefficient on certain special cases. As a test case, they used a file containing 1000 blank lines, and a second file, consisting of the initial file, but containing a single non-blank file on both ends. As a fact, using other experiments performed on typical files, \emph{Miller's} and \emph{Myers'} algorithm ran roughly four times faster \cite[p. 1034f]{miller1985file}.

\subsection{Technology}
\label{sec:diff-technology}

\begin{lstlisting}[label={list:diff-normalformat}, caption=sample.diff]
0a1
> w
3,4c4,6
< c
< d
---
> x
> y
> z
6,7d7
< f
< g
\end{lstlisting}

\texttt{diff's} core task is finding the ``shortest sequence of insertions and deletions that will convert the first string to the second'' \cite[1025]{miller1985file} together with finding the longest common subsequence occurring in both files \cite[2]{Hunt1976}. Together with the mathematical algorithm, it should provide an easily understandable format for humans, consisting of line numbers combined with \emph{a, c} or \emph{d} (append, change, delete), as well as \emph{<} and \emph{>} line prefixes, showing the affiliation either to the initial or compared file. This is called the ``Normal Format'' \cite[12]{mackenzie2003comparing}. Listing \ref{list:diff-normalformat} shows a sample output, comparing the strings \texttt{a b c d e f g} and \texttt{w a b x y z e} (one line per letter)\cite[p. 1f]{Hunt1976}.


% TODO: Ugly, remove if not necessary anymore, cares for vertical spaces above subsections
\vspace{20pt}
\subsubsection{The Unified Format}

To provide a more readable user experience, \emph{GNU diff} contains an improved format, called \emph{Unified Format}, removing redundancy by using a more compact syntax. It can be selected as output format by executing \texttt{diff} together with a \texttt{-u} flag \cite[16]{mackenzie2003comparing}, whereas \texttt{git diff} uses this as standard format to show changes within the current working tree\footnote{\url{https://git-scm.com/docs/git-diff} -- git diff documentation on \texttt{Git's} website}.


\begin{lstlisting}[label={list:diff-unifiedformat}, caption=unified\_format.diff]
--- oldfile	2017-04-13 09:42:47.474769553 +0200
+++ newfile	2017-04-13 09:43:13.898566935 +0200
@@ -1,7 +1,7 @@
+w
 a
 b
-c
-d
+x
+y
+z
 e
-f
-g
\end{lstlisting}

As an example, Listing \ref{list:diff-unifiedformat} shows the same \texttt{diff} output as Listing \ref{list:diff-normalformat}, only as \emph{Unified Format} using the following components:

\begin{description}
  \item[\texttt{------ \{filename\} \{timestamp\}}] -- indicates the initial file together with the timestamp it was created,
  \item[\texttt{+++ \{filename\} \{timestamp\}}] -- same as above, but for the compared file
  \item[\texttt{@@ -\{intial line range\} +\{compared line range\} @@}] -- \texttt{-1,7} indicates the following 7 lines, starting from the first line of the initial file
  \item[\texttt{+}] -- marks a line as added in compared file
  \item[\texttt{-}] -- marks a line as deleted in compared file
\end{description}

To make the above explanation a little bit more clear, an additional example with a text speaking for itself is added below:

\lstinputlisting[caption={file.diff}, label={list:diff-explanation}]{chapters/03-technical-foundations/_support/file.txt}

It can be clearly seen, that line 3 shows a growth of \emph{newfile} by two lines: \texttt{-1,8} vs. \texttt{+1,10}. Having also a color-coded representation, it would boost the readability of such \texttt{diff} outputs once again.
