\chapter{Technical foundations}
\label{cha:technicalfoundations}

Before any explanation of the theoretical approach behind this research, there is a need of describing the technical foundations, on which the general thoughts were built on.

Being mostly a back-end web developer since the start of my studies, I experienced a lot of changes in this field. Changes which were mostly caused by the ongoing progression of general web development, but also caused by constantly coming and going ``hypes''. Changes which mostly left minor traces, but sometimes had a major impact on the way I process things during my work on different projects.

\paragraph{EcmaScript 6.}
One of these major turning points was the introduction of EcmaScript 6 (\texttt{ES6}), which I first heard from in a talk given by \emph{Douglas Crockford}\footnote{\url{http://crockford.com} -- Douglas Crockford's website.} in March 2015. He is -- besides other projects -- not only the creator of \texttt{JSLint}\footnote{\url{http://www.jslint.com/help.html} -- JSLint, a code quality tool.}, but also the author of ``\emph{The Good Parts}'' \cite{crockford2008javascript}, a language reference for \texttt{JavaScript}.

\texttt{ES6} introduced a lot of new features, pushing \texttt{JavaScript} more and more towards the definition of an ``object-oriented'' scripting language, thus not only by providing ``real classes'', instead of the old and cumbersome approach of inheritence by setting a constructor function into the \texttt{prototype} object \cite[47]{crockford2008javascript}.

However, the probably most beneficial functions are \texttt{Promises}\footnote{\url{https://developer.mozilla.org/en-US/docs/Web/JavaScript/Reference/Global_Objects/Promise} -- Promise reference on the Mozilla Developer Network.} and \texttt{Arrow functions}\footnote{\url{https://developer.mozilla.org/en-US/docs/Web/JavaScript/Reference/Functions/Arrow_functions} -- Arrow functions reference on the Mozilla Developer Network.}, both saving significant amounts of code -- especially when working with asynchronous environments, like HTTP requests.

\paragraph{Static site generation.}
Ever since I first got to work with a static site generator, I got more and more interested in how easy and fast static site projects may be scaffolded. Once completely set up, I was simply impressed by its power and variety -- this differed completely from the dynamic web environments I used to work with at that time.

A dynamic web project often demands lots of preparation before receiving any visible outcome, sometimes even when using a predefined framework. Additionally, though many projects in the \texttt{Node.js} universe are already matured to an extent, where they may be even used for enterprise projects, there is still a remaining risk for failing a client due to the amount of dependencies on external services like databases, session storages or user management tools.

\section{Build pipelines}
\label{sec:buildpipelines}

%% Graphic of build pipeline components
\begin{figure} % h-ere, t-op, b-ottom, p-age
    \centering
    \includegraphics[width=0.9\textwidth]{build_pipeline.png}
    \caption{A graphic showing the basic flow of a \emph{build pipeline}.\\
    First, the configuration file is read and necessary modules invoked. Second, the global metadata, as well as metadata from within the content, is parsed and stored in the global configuration. Third, the content sources get compiled into basic \emph{HTML} markup. Last, the compiled content gets rendered into predefined templates, to apply a given structure which is used commonly throughout the website.}
    \label{fig:build-pipeline}
\end{figure}
%

A static site generator mostly consists of a build pipeline, which handles the workflow needed for bringing the content into shape. This goes from setting the boundaries, determined by a configuration file, to finally producing a web root, consisting of HTML, CSS and JavaScript files, as well as images.

Normally, the major part of it happens sequentially, as nearly all content files are facing a series of transformations on them \cite{Metalsmith2015technicaldocumentation}. Although the amount and extent of conversions may differ significantly from pipeline setup to pipeline setup, it can be broken down to the following core parts (see fig. \ref{fig:build-pipeline}):

\begin{description}
  \item[Metadata parser] -- Parses global metadata, found in the configuration file or in the YAML frontmatter of content files.
  \item[Markdown compiler] -- Used to convert easily read- and writeable Markdown files in browser-readable HTML.
  \item[Template renderer] -- Responsible for bringing the very basic content structure in shape. The goal should be a common appearence, enriched with additional elements (like navigation, breadcrumbs, \ldots).
\end{description}

Of course, the list above overlaps at some point with the list mentioned in <<\emph{\nameref{par:creatingcontent}}>> on p. \pageref{par:creatingcontent}, as a build pipeline may be considered only as a part of the given static site generator (although the main part), not as the generator itself. One of the reasons is its independence of programming languages: A build pipeline doesn't care which programming language it consists of, as long as it knows how to interpret the content sources and templates. Therefore, it merely should be called a concept, not a framework.

%% Say something about the config file, yadda yadda yadda one hand pure build pipeline, because of metadata and configuration, other hand no, because of plugins. plugins??

\subsubsection{Frontmatter}
\label{sec:buildpipelines-frontmatter}

\lstinputlisting[caption=frontmatter-demo.md]{chapters/03-technical-foundations/_support/frontmatter.md}
\label{list:frontmatter}

Listing \ref{list:frontmatter} shows a sample usage of \texttt{frontmatter} inside a \texttt{Markdown} file. Bounded by three dashes above the main content source, it allows certain per-file metadata definitions, which will be parsed at build time and provided for the template rendering engine.
% Mention something about omitting databases using frontmatter meta

As an example, the selected template for this sample file may also hold a list for the mentioned \emph{tags}, as well as a placeholder for the \emph{author}'s name and/or \emph{title}. The main content gets then rendered into the respective placeholding tag, already self-containing a basic structure.

Using a \emph{template: false} declaration, some plugins may prevent rendering the content into a template. This might be interesting in cases, where different partials should be included in the DOM by some sort of asynchronous \texttt{JavaScript} later on.


% TODO: Ugly, remove if not necessary anymore, cares for vertical spaces above subsections
\vspace{20pt}

\subsection{Markdown}
\label{sec:buildpipelines-markdown}

\lstinputlisting[caption={markdown.md}, label={list:markdown-demo}]{chapters/03-technical-foundations/_support/markdown.md}

Markdown consists of shorthand conventions, which should be easier to type for content creators  \cite[38]{dhillon2016}.
Therefore, it makes understanding HTML not a necessary precondition anymore, as a basic content structure may be easily achieved when prepending/surrounding text with certain special characters like \texttt{\#, *, \_, \ldots} (See Listing \ref{list:markdown-demo}).

Originally created by \emph{John Gruber} as a plugin for \emph{Movable Type} and \emph{Blosxom} blogging engines in March 2004 \cite{Markdown2004introduction}\cite{Markdown2004main}, it should be a supportive tool for users against the complexity of formal markup languages (e.g. HTML5) \cite[4]{RFC7764}. According to Gruber's intention, there is no ``invalid'' \texttt{Markdown}, as he suggests the author should either ``keep on experimenting'' or ``change the processor'', if the output happens to fail his/her expectations \cite[5]{RFC7764}.

GitHub finally adapted Markdown to its own version, called ``GitHub Flavored Markdown'' (\emph{GFM}), somewhere around April 2009\footnote{\url{https://github.com/mojombo/github-flavored-markdown/issues/1} -- GitHub Flavored Markdown examples by Tom Preston-Werner.}. The people behind it enhanced its original functions to also support \emph{code blocks, tables, strike-through text}, as well as \emph{auto-linking} URL structures within the content \cite[18]{RFC7764}. Additionally, also GitHub-specific functions, such as \emph{user mentions, commit references} or \emph{emojis}, may be used \cite{GithubFlavoredMarkdown}.

Since then, GitHub renders browser-friendly versions of general descriptions written in Markdown, for providing fast and easy overviews of the respective repository. As an example, an existing \texttt{README.md} always appears below the root file tree section on a repository front page \cite[5]{gandrud2013github}.


% TODO: Ugly, remove if not necessary anymore, cares for vertical spaces above subsections
\vspace{20pt}

\subsection{Templates}
\label{sec:buildpipelines-templates}

\emph{Templates} are the frames of each content page, caring for a common, browser-readable HTML structure. A uniform design layout allows site-wide look and feel using a global CSS style sheet, as well as certain events triggered by user behaviour, handled by a single JavaScript file.

Born out of the need to fail-safe produce HTML on the server, as the produced data chunks -- initiated by a client HTTP request -- steadily grew, the goal behind templating engines is primarily to separate business logic from data display. Ideally, at the end of the day, there should be no code in HTML, and no HTML in code \cite[225]{Parr2004templates}.

\begin{program}
  \caption{post.hbs}
  \label{list:handlebars}
\lstinputlisting[language=HTML]{chapters/03-technical-foundations/_support/post.html}
\end{program}

A simple template file example for a blog post is shown in Program \ref{list:handlebars}. It is written in \emph{Handlebars}, a very basic templating language, offering only a very limited amount of template logic. Included are \emph{loops, if/else, partials,~\ldots}~--~however, additional ``\emph{helper}''-functions may be added by the developer.

Although such template logic may come in handy for the most part, as some business logic decisions seem to be rather taken during rendering, instead of being hard-coded before, the concept of data separation is therefore often unknowingly violated \cite[228]{Parr2004templates}. Thus, the choice of the ``optimal'' templating engine for a given project is crucial, as different engines offer a different range of built-in logic. This could go from very conservative \emph{Mustache}\footnote{\url{https://mustache.github.io/} -- Mustache website.} to very powerful ones like \emph{Liquid}\footnote{\url{https://help.shopify.com/themes/liquid} -- Shopify's Liquid template engine.} (see Sec. \ref{sec:jekyll-technology}) or \emph{EJS}\footnote{\url{http://www.embeddedjs.com/} -- Embedded JavaScript website}.


\section{Git}
\label{sec:git}

Today, hardly any software project is started without any form of version control system (\texttt{VCS}). It supports developers as a back-up system and living archive of their work, as data generally is ephemeral and can be lost easily \cite[1]{loeliger2012version}. Although there are many different variations offered, some of the most popular today are \emph{Git, Apache Subversion, Mercurial and Bazaar}.

\subsection{History}
\label{sec:git-history}
\texttt{Git} was initially published by \emph{Linus Torvalds} on April 7\textsuperscript{th}, 2005 \cite[6]{loeliger2012version}. This was necessary, as the ``free'' version of the then used VCS for the \emph{Linux} kernel development, \texttt{BitKeeper}, was restricted in a way it was not suitable any longer for the community behind it. Furthermore, the search of an already available alternative to the \texttt{BitKeeper} system failed due to an unsatisfying combination of needed features, so \emph{Torvalds} came up with his own VCS flavor, containing all desired functionalities for further \emph{Linux} development (among others) \cite[4]{loeliger2012version}:

\begin{itemize}
  \item Distributed development
  \item Handle thousands of developers
  \item Efficient performance
  \item Support branched development
  \item Free, as in freedom
\end{itemize}

%% Write something, why it is so famous. distributed development - other use cases .. blargh.

\section{Diff}
\label{sec:diff}

``\emph{diff} reports file differences between two files, expressed as a minimal list of line changes (\ldots)'' \cite[1]{Hunt1976}. Existing more than 40 years now, it has been an essential tool for file comparison throughout the history of computing -- furthermore, it is also a core component of Git, which contains its own version called \emph{git diff} \cite[108]{loeliger2012version}.

\subsection{History}
\label{sec:diff-history}
Initially published by \emph{James W. Hunt} and \emph{Malcolm D. McIlroy} in July 1976 when working at \emph{Bell Labs}, the algorithm was later used in \emph{UNIX} as application called diff. \emph{Paul Eggert} and \emph{Richard Stallman} (among others) also wrote the diff application as part of their \emph{GNU diffutils}\footnote{\url{http://manpages.ubuntu.com/manpages/zesty/en/man1/diff.1.html} -- Manpage for GNU diff.}, which is nowadays mainly distributed in Linux derivatives, MacOS, as well as part of Git. They used an improved algorithm published by \emph{Webb Miller} and \emph{Eugene W. Myers} in 1985 \cite[3]{mackenzie2003comparing}, who proved that the original ``Hunt-McIlroy algorithm'' is inefficient on certain special cases. As a test case, they used a file containing 1000 blank lines, and a second file, consisting of the initial file, but containing a single non-blank file on both ends. As a fact, using other experiments performed on typical files, Miller's and Myers' algorithm ran roughly four times faster \cite[p. 1034f]{miller1985file}.

\subsection{Technology}
diff's core task is finding the ``shortest sequence of insertions and deletions that will convert the first string to the second'' \cite[1025]{miller1985file} together with finding the longest common subsequence occurring in both files \cite[2]{Hunt1976}. Combined with the mathematical algorithm, it should provide an easily understandable format for humans, consisting of line numbers joined with \emph{a, c} or \emph{d} (append, change, delete), as well as \emph{<} and \emph{>} line prefixes, showing the affiliation either to the initial or compared file. This is called the ``Normal Format'' \cite[12]{mackenzie2003comparing}. Program \ref{list:diff-normalformat} shows a sample output, comparing the strings \texttt{a b c d e f g} and \texttt{w a b x y z e} (one line per letter) \cite[p. 1f]{Hunt1976}.

\label{sec:diff-technology}
\begin{program}
  \caption{sample.diff}
  \label{list:diff-normalformat}
  \begin{GenericCode}
  0a1
  > w
  3,4c4,6
  < c
  < d
  ---
  > x
  > y
  > z
  6,7d7
  < f
  < g
  \end{GenericCode}
\end{program}

% TODO: Ugly, remove if not necessary anymore, cares for vertical spaces above subsections
% \vspace{20pt}

\subsubsection{The Unified Format}
To provide a more readable user experience, GNU diff contains an improved format, called Unified Format, removing redundancy by using a more compact syntax. It can be selected as output format by executing diff together with a \texttt{-u} flag \cite[16]{mackenzie2003comparing}, whereas git diff uses this as standard format to show changes within the current working tree \cite{GitDiff}.

\begin{program}
  \caption{unified\_format.diff}
  \label{list:diff-unifiedformat}
  \begin{GenericCode}
  --- oldfile	2017-04-13 09:42:47.474769553 +0200
  +++ newfile	2017-04-13 09:43:13.898566935 +0200
  @@ -1,7 +1,7 @@
  +w
   a
   b
  -c
  -d
  +x
  +y
  +z
   e
  -f
  -g
  \end{GenericCode}
\end{program}

As an example, Program \ref{list:diff-unifiedformat} shows the same diff output as Program \ref{list:diff-normalformat}, only as Unified Format using the following components: \texttt{------ \{filename\} \{timestamp\}} indicates the initial file together with the timestamp it was created, whereas \texttt{+++ \{filename\} \{timestamp\}} is the same as above, but for the compared file.
\texttt{@@ -\{intial line range\} +\{compared line range\} @@} shows the affected line range, where \texttt{-1,7} indicates the following 7 lines, starting from the first line of the initial file. Lastly \texttt{+} marks a line as added in compared file and \texttt{-} marks a line as deleted in the compared file. To make the above explanation a little bit more clear, an additional example with a text speaking for itself is shown in Program \ref{list:diff-explanation}.

\begin{program}
  \caption{file.diff}
  \label{list:diff-explanation}
\lstinputlisting{chapters/03-technical-foundations/_support/file.txt}
\end{program}

It can be clearly seen, that line 3 shows a growth of \emph{newfile} by two lines: \texttt{-1,8} vs. \texttt{+1,10}. Having also a color-coded representation, it would boost the readability of such diff outputs once again.


% TODO: Ugly, remove if not necessary anymore, cares for vertical spaces above subsections
%\vspace{20pt}
\subsubsection{Usage with Git}
As already stated, diff is one of the core components of Git. Not only does it support determining changes in the source code between a snapshot and another, it may also reveal merge conflicts, if segments are mutually exclusive and therefore preventing a flawless propagation of development. Thus, a varying development history of different origins (e.g. branches) not compatible to each other might be indicated. Furthermore, a conflict may also happen, if a developer forgot to \emph{pull} the latest changes before committing his/her current development progress. These conflicts may only be handled through human guidance \cite[124]{loeliger2012version}.

\begin{program}
  \caption{A snippet of a file called ``manual.txt'', which is affected by a conflict. Content between \texttt{HEAD} and \texttt{=======} contains the local version, content below contains the foreign conflicting version.}
  \label{list:diff-conflict}
\begin{GenericCode}
<<<<<<< HEAD:manual.txt
I (the developer) am right!
=======
The branch_name is right!
>>>>>>> branch_name:manual.txt
\end{GenericCode}
\end{program}

A conflict presents itself primarily through a message similar to:\\
\texttt{CONFLICT (content): Merge conflict in file\\
Automatic merge failed; fix conflicts and then commit the result.}\\
If anything like the above happens, the affected files by the conflict also contain a structure like shown in Program \ref{list:diff-conflict}. A conflict can then be resolved by removing its markers and picking the appropriate resolution of either side of the \texttt{=======} delimiter \cite{GitConflicts}, as well as mixing them to the developers needs \cite[126]{loeliger2012version}. A single file may also contain multiple conflicts.

\subsubsection{Usage with GitHub}
Especially when interacting with GitHub's REST API, it is very easy to generate and import file diffs of a given repository. Whether two branches or two commits using their \emph{SHA} values are compared, a single HTTP request suffices for programmatically retrieving data, which is normally only accessible using a terminal emulator.

Depending on the requested \emph{media type} in the appropriate HTTP header field, either a full-featured diff, patch or JSON containing per-file patches is emitted by the API. If the latter was used, the underlying diff is translated into a JSON object, containing information like the number of additions, deletions and changes, as well as the mentioned \emph{patch} for each file.

As a consequence, a repository does not necessarily have to be \emph{cloned}, as it may be patched constantly using the API to keep it up to date -- using this method, patch is even able to create and delete files, if necessary \cite[57]{mackenzie2003comparing}. The only prerequisite is to keep track of the single commit hashes the patches are applied from.

