\chapter{Kurzfassung}

\begin{german}
Während moderne Content Management Systeme ein gut entwickeltes Umfeld für Online-Inhaltsverwaltung und -erstellung bereitstellen, sind sie doch auf eine Anzahl externer Services angewiesen. Dazu zählen vordergründig Datenbanksysteme, aber auch verschiedene Login-Mechanismen, um Zugang zu einem gesperrten Editierbereich freizugeben. Durch die Abhängigkeit von derartigen Services entsteht bei zunehmender Größe des jeweiligen Projekts ein Anstieg des Aufwands für die Verwaltung dieser Erweiterungen.

Static Site Generatoren andererseits benötigen keine externen Erweiterungen, da ihre einzige Aufgabe darin besteht, die Website-Quellcodes in eine für Webbrowser lesbare Version zu konvertieren. Allerdings haben diese Static Site Generatoren einen erheblichen Nachteil; da sie nicht zwischen bereits vorhandenen und neuen Inhalten unterscheiden können, ist jedesmal ein vollständiger Neubau der Website-Quellen notwendig.

Diese Masterarbeit soll daher einen Lösungsweg für dieses Problem aufzeigen. Durch einen selektiven Algorithmus sollen am Ende nur die wirklich notwendigen Inhalte gebaut werden und in eine vorhandene Dateistruktur eingebunden werden. Zusammen mit einer REST API soll zusätzlich eine benutzerfreundliche Interaktion und verbesserte Arbeitsteilung möglich sein.
\end{german}
